\documentclass[a4paper,pagesize=pdftex,DIV=16]{scrartcl}
\usepackage{indentfirst}

\usepackage[width=21cm,
height=29.7cm,
top=2cm,
inner=2cm,
outer=2cm,
margin=2cm]{geometry}
\usepackage{xcolor}

\usepackage[hidelinks]{hyperref}

\newlength\TextFieldLength
\newcommand\TextFieldFill[2][]{%
	\setlength\TextFieldLength{\linewidth}%
	\settowidth{\dimen0}{#2 }%
	\addtolength\TextFieldLength{-\dimen0}%
	\addtolength\TextFieldLength{-2.22221pt}%
	\TextField[#1,width=\TextFieldLength]{\raisebox{2pt}{#2 }}%
}
\renewcommand*\footnoterule{}

\begin{document}
\begin{center}
	\Large\textbf{ATTESTATION DE DÉPLACEMENT DÉROGATOIRE}
\end{center}

En application de l’article 3 du décret du 23 mars 2020 prescrivant les mesures générales nécessaires pour faire face à l’épidémie de Covid19 dans le cadre de l’état d’urgence sanitaire\\


\setlength\parindent{0pt}
Je soussigné(e),\\

\begin{Form}
    \TextFieldFill[backgroundcolor=gray!30,bordercolor=,height=1.4em]{Mme/M. : } \\[1.3ex]
    \TextFieldFill[backgroundcolor=gray!30,bordercolor=,height=1.4em]{Né(e) le : } \\[1.3ex]
    \TextFieldFill[backgroundcolor=gray!30,bordercolor=,height=1.4em]{\`A : } \\[1.3ex]
    \TextFieldFill[backgroundcolor=gray!30,bordercolor=,height=1.4em]{Demeurant : } \\
\end{Form}

certifie que mon déplacement est lié au motif suivant (cocher la case) autorisé par l’article 3 du décret du 23 mars 2020 prescrivant les mesures générales nécessaires pour faire face à l’épidémie de Covid19 dans le cadre de l’état d’urgence sanitaire\footnotemark[1] :\\

\footnotetext[1]{Les personnes souhaitant bénéficier de l'une de ces exceptions doivent se munir s’il y a lieu, lors de leurs déplacements hors de leur domicile, d'un document leur permettant de justifier que le déplacement considéré entre dans le champ de l'une de ces exceptions.}
\def\LayoutCheckField#1#2{% label, field
	#2 #1%
}

\begin{minipage}{0.05\textwidth}
	\begin{Form}
		\CheckBox[backgroundcolor=gray!30,width=1.5em,bordercolor=black,height=1.5em]{\ \unskip}\ 
	\end{Form}
\end{minipage}
\begin{minipage}{0.94\textwidth} Déplacements entre le domicile et le lieu d’exercice de l’activité professionnelle, lorsqu’ils sont indispensables à l’exercice d’activités ne pouvant être organisées sous forme de télétravail ou déplacements professionnels ne pouvant être différés.\footnotemark[2]
\end{minipage}
\footnotetext[2]{\`A utiliser par les travailleurs non-salariés, lorsqu’ils ne peuvent disposer d’un justificatif de déplacement établi par leur employeur}
\\[1.8ex]
\begin{minipage}{0.05\textwidth}
	\begin{Form}
		\CheckBox[backgroundcolor=gray!30,width=1.5em,bordercolor=black,height=1.5em]{\ \unskip}\ 
	\end{Form}
\end{minipage}
\begin{minipage}{0.94\textwidth} Déplacements pour effectuer des achats de fournitures nécessaires à l’activité professionnelle et des achats de première nécessité\footnotemark[3] dans des établissements dont les activités demeurent autorisées (liste sur gouvernement.fr).
\end{minipage}
\footnotetext[3]{Y compris les acquisitions à titre gratuit (distribution de denrées alimentaires\ldots) et les déplacements liés à la perception de prestations sociales et au retrait d’espèces.}
\\[1.8ex]
\begin{minipage}{0.05\textwidth}
	\begin{Form}
		\CheckBox[backgroundcolor=gray!30,width=1.5em,bordercolor=black,height=1.5em]{\ \unskip}\ 
	\end{Form}
\end{minipage}
\begin{minipage}{0.94\textwidth} Consultations et soins ne pouvant être assurés à distance et ne pouvant être différés ; consultations et soins des patients atteints d'une affection de longue durée.
\end{minipage}
\\[1.8ex]
\begin{minipage}{0.05\textwidth}
	\begin{Form}
		\CheckBox[backgroundcolor=gray!30,width=1.5em,bordercolor=black,height=1.5em]{\ \unskip}\ 
	\end{Form}
\end{minipage}
\begin{minipage}{0.94\textwidth} Déplacements pour motif familial impérieux, pour l’assistance aux personnes vulnérables ou la garde d’enfants.
\end{minipage}
\\[1.8ex]
\begin{minipage}{0.05\textwidth}
	\begin{Form}
		\CheckBox[backgroundcolor=gray!30,width=1.5em,bordercolor=black,height=1.5em]{\ \unskip}\ 
	\end{Form}
\end{minipage}
\begin{minipage}{0.94\textwidth} Déplacements brefs, dans la limite d'une heure quotidienne et dans un rayon maximal d'un kilomètre autour du domicile, liés soit à l'activité physique individuelle des personnes, à l'exclusion de toute pratique sportive collective et de toute proximité avec d'autres personnes, soit à la promenade avec les seules personnes regroupées dans un même domicile, soit aux besoins des animaux de compagnie.
\end{minipage}
\\[1.8ex]
\begin{minipage}{0.05\textwidth}
	\begin{Form}
		\CheckBox[backgroundcolor=gray!30,width=1.5em,bordercolor=black,height=1.5em]{\ \unskip}\ 
	\end{Form}
\end{minipage}
\begin{minipage}{0.94\textwidth} Convocation judiciaire ou administrative. 
\end{minipage}
\\[2.1ex]
\begin{minipage}{0.05\textwidth}
	\begin{Form}
		\CheckBox[backgroundcolor=gray!30,width=1.5em,bordercolor=black,height=1.5em]{\ \unskip}\ 
	\end{Form}
\end{minipage}
\begin{minipage}{0.94\textwidth} Participation à des missions d’intérêt général sur demande de l’autorité administrative.
\end{minipage}
\\[2ex]

\begin{Form}
	\TextFieldFill[backgroundcolor=gray!30,bordercolor=,height=1.4em]{Fait à : } \\[1.3ex]
	\TextField[backgroundcolor=gray!30,bordercolor=,width=3cm,height=1.4em]{Le : } \TextField[backgroundcolor=gray!30,bordercolor=,width=1cm,height=1.4em]{à}
	\TextField[backgroundcolor=gray!30,bordercolor=,width=1cm,height=1.4em]{h}\\
	(Date et heure de début de sortie à mentionner obligatoirement)
	 \\[1.5ex]
	\TextFieldFill[backgroundcolor=gray!30,bordercolor=,height=1.4em]{Signature : } \\
\end{Form}
\end{document}